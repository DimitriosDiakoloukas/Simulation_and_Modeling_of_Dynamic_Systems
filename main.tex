\documentclass[a4paper,12pt]{report}

\usepackage{ucs}
\usepackage[utf8x]{inputenc} % Input encoding for Greek characters
\usepackage[greek,english]{babel} % Language support

\newcommand{\en}{\selectlanguage{english}}
\newcommand{\gr}{\selectlanguage{greek}}

% \usepackage{algorithm2e}
% \usepackage{algorithm}
% \usepackage{algorithmic}
\usepackage{enumitem}
\usepackage{float}
\usepackage{amsmath}
\usepackage{graphicx} % For including images
\usepackage{titlesec} % Custom title formatting
\usepackage{fancyhdr} % For custom headers and footers
\usepackage{geometry} % For adjusting page margins

% Adjust the page margins to make content wider
\geometry{top=2.5cm, bottom=2.5cm, left=2.5cm, right=2.5cm}

% Redefine chapter formatting to make it smaller
\titleformat{\chapter}[display]
    {\normalfont\LARGE\bfseries} % Smaller size and bold for chapter heading
    {\chaptername\ \thechapter} % Chapter number format
    {15pt} % Space between chapter number and title
    {\bfseries} % Smaller size and bold for chapter title
\begin{document}

\begin{titlepage}
    \centering
    \vspace*{-3cm}
    % University logo
    \includegraphics[width=1\textwidth]{auth_logo.png} % Replace with your actual logo file

    % University name in Greek
    \textbf{\gr ΑΡΙΣΤΟΤΕΛΕΙΟ ΠΑΝΕΠΙΣΤΗΜΙΟ ΘΕΣΣΑΛΟΝΙΚΗΣ}
    \vspace{2cm}

    % Document title and subtitle in Greek
    \LARGE\textbf{\gr Προσομοίωση και Μοντελοποίηση Δυναμικών Συστημάτων Αναφορά} \\
    \Large\normalfont{\gr Εργασία 1} \\
    \vspace{4cm}

    \gr
    \large
    \textbf{Διακολουκάς Δημήτριος} \\
    \textbf{AEM 10642}
    \vspace{2.5cm}

    \en
    \textit{Email: ddiakolou@ece.auth.gr}
\end{titlepage}

\gr
\tableofcontents

\chapter{\gr Ανάλυση προβλήματος και εισαγωγή}
\label{chap:intro}

Στην εργασία αυτή μελετούμε ένα απλό εκκρεμές με ροπή εισόδου. 
Η \emph{γραμμικοποιημένη} εξίσωση κίνησης, για μικρές γωνίες $q(t)$ 
(με την προσέγγιση $\sin(q)\approx q$), δίνεται από την ακόλουθη διαφορική εξίσωση:
\begin{equation}
m L^2\,\ddot q(t)\;+\;c\,\dot q(t)\;+\;m g L\,q(t)
\;=\;
u(t),
\tag{1}
\end{equation}
όπου:
\begin{itemize}
\item $q(t)\,[\text{\en rad\gr}]$ είναι η γωνία εκτροπής του εκκρεμούς,
\item $m\,[\text{\en kg\gr}]$ η μάζα,
\item $L\,[\text{\en m\gr}]$ το μήκος του εκκρεμούς,
\item $c\,[\text{\en N\gr}\cdot\text{\en s\gr}]$ ένας σταθερός συντελεστής απόσβεσης,
\item $g\,[\text{\en m\gr}/\text{\en s\gr}^2]$ η επιτάχυνση της βαρύτητας,
\item $u(t)\,[\text{\en N\gr}\cdot\text{\en m\gr}]$ είναι η ροπή/είσοδος ελέγχου.
\end{itemize}

\paragraph{Στόχος:}
Η επιδίωξη είναι να \emph{μελετήσουμε} και \emph{υποβιβάσουμε} \en(linearize\gr/αναλύσουμε)
το σύστημα, ξεκινώντας από την εξίσωση (1), ώστε να προκύψει κατάλληλο \emph{μοντέλο 
κατάστασης} και, στην συνέχεια, να εφαρμοστούν τεχνικές προσομοίωσης ή/και ελέγχου. 
Συγκεκριμένα, θα εστιάσουμε στο πώς μπορούμε να εκφράσουμε το σύστημα σε 
\emph{μορφή καταστάσεων}, δηλαδή:
\[
\dot{x}(t) \;=\; A\,x(t)\;+\;B\,u(t),
\quad
y(t) \;=\;C\,x(t),
\]
και πώς να υλοποιήσουμε τη σχετική προσομοίωση και ανάλυση.

\bigskip

\chapter{Εξισώσεις κατάστασης και προσομοίωση απόκρισης συστήματος}

\section{Θέμα 1(α) Σχηματισμός Εξισώσεων Κατάστασης}

Σύμφωνα με την εκφώνηση του Θέματος~1(α), ζητείται να γραφούν οι εξισώσεις κατάστασης 
στη μορφή $\dot{x}(t) = A\,x(t) + B\,u(t)$. Ορίζουμε ως \emph{διάνυσμα κατάστασης}:
\[
x(t) 
\;=\;
\begin{bmatrix}
q(t) \\
\dot{q}(t)
\end{bmatrix}.
\]
Τότε, χρησιμοποιώντας την (1), παρατηρούμε ότι:
\[
\dot{x}_1(t) = \dot{q}(t) = x_2(t),
\]
\[
\dot{x}_2(t) 
= \ddot{q}(t)
= \frac{1}{m L^2}\Bigl(u(t) \;-\; c\,\dot{q}(t)\;-\;m g L\,q(t)\Bigr).
\]
Συνδυάζοντας τα παραπάνω, λαμβάνουμε:
\[
\dot{x}(t)
\;=\;
\underbrace{\begin{bmatrix}
0 & 1\\
-\frac{m g L}{m L^2} & -\frac{c}{m L^2}
\end{bmatrix}}_{A}
\,x(t)
\;+\;
\underbrace{\begin{bmatrix}
0 \\[2pt]
\frac{1}{m L^2}
\end{bmatrix}}_{B}
\,u(t).
\]
Στην περίπτωση που η έξοδος $y(t)$ οριστεί ως η γωνία $q(t)$, έχουμε:
\[
y(t) 
\;=\;
\underbrace{\begin{bmatrix} 1 & 0 \end{bmatrix}}_{C}\,x(t).
\]

\hspace{-0.6cm}Με αυτόν τον τρόπο επιτυγχάνεται ο \emph{σχηματισμός του μοντέλου κατάστασης} του 
εκκρεμούς:  
\[
\dot{x}(t) \;=\; A\,x(t) \;+\; B\,u(t),
\qquad
y(t) \;=\;C\,x(t).
\]

\bigskip

\section{Θέμα 1(β) Προσομοίωση \emph{\en (ODE Solver \gr στο \en MATLAB)\gr}}

Στην παράγραφο αυτή, εφαρμόζουμε κάποια συνάρτηση επίλυσης διαφορικών εξισώσεων (π.χ. \en \texttt{ode45}\gr) του \en \textsc{MATLAB}\gr, ώστε να βρούμε την απόκριση του συστήματος για:
\begin{itemize}
\item Μηδενικές αρχικές συνθήκες, δηλαδη: $q(0)=0$, $\dot{q}(0)=0$,
\item Είσοδο (ροπή) της μορφής $u(t)=A_0 \sin(\omega t)$, $\forall t \ge 0$,
\item Παράμετροι: $m=0.7$, $L=1.25$, $c=0.15$, $g=9.81$, $A_0=4$, $\omega=2$,
\item Χρονικό ορίζοντα προσομοίωσης $[0,\,20]\,\mathrm s$,
\item Βήμα ολοκλήρωσης περίπου $\Delta t \approx 10^{-3}\,\mathrm s$ για \emph{ακριβή} αποτέλεσμα.
\end{itemize}

\subsection*{Μαθηματική Ερμηνεία της Επίλυσης με \en \texttt{ode45}\gr}

Στην προηγούμενη ενότητα είδαμε την υπολογιστική επίλυση του γραμμικοποιημένου
εκκρεμούς χρησιμοποιώντας τη συνάρτηση \en \texttt{ode45} \gr στο \en \textsc{MATLAB}\gr.
Ακολουθεί μια \emph{μαθηματική} ερμηνεία του ίδιου αλγορίθμου, χωρίς να
παραθέτουμε τον ίδιο τον κώδικα:

\begin{enumerate}
    \item \textbf{Διαφορική εξίσωση εκκρεμούς:} \ 
          Αξίζει να σημειωθεί ότι η βασική εξίσωση κίνησης, γραμμικοποιημένη για μικρές
          γωνίες, είναι:
          \begin{equation}
          \ddot{q}(t) \;+\; \frac{c}{mL^2}\,\dot{q}(t) \;+\; \frac{g}{L}\,q(t)
          \;=\;
          \frac{1}{mL^2}\,u(t).
          \tag{2}
          \label{eq:pendulum-lin}
          \end{equation}
          Με $m=0.7\,\mathrm{kg}$, $L=1.25\,\mathrm{m}$, $c=0.15\,\mathrm{N\cdot m\,s}$,
          $g=9.81\,\mathrm{m/s^2}$ και μια δύναμη/ροπή εισόδου $u(t)=A_{0}\sin(\omega t)$.

    \item \textbf{Μορφή κατάστασης:} \ 
          Για να εφαρμόσουμε αριθμητικό αλγόριθμο επίλυσης όπως το \en \texttt{ode45}\gr,
          περνάμε από την τυπική διαφορική εξίσωση 2ης τάξης στη μορφή συστήματος
          1ης τάξης. Ορίζουμε
          \[
          x_{1}(t) \;=\; q(t),
          \quad
          x_{2}(t) \;=\; \dot{q}(t),
          \]
          οπότε η εξίσωση~\eqref{eq:pendulum-lin} γράφεται ως
          \[
          \dot{x}_{1}(t) \;=\; x_{2}(t),
          \]
          \[
          \dot{x}_{2}(t) 
          \;=\; \frac{1}{mL^{2}}\,\Bigl[\,u(t)\;-\;c\,x_{2}(t)\;-\;m\,g\,L\,x_{1}(t)\Bigr].
          \]
          Με αυτόν τον τρόπο έχουμε ένα σύστημα δύο διαφορικών εξισώσεων 1ης τάξης.

    \item \textbf{Αρχικές συνθήκες και είσοδος:} \ 
          Στην περίπτωσή μας, θέτουμε $x_{1}(0)=q(0)=0$ και $x_{2}(0)=\dot{q}(0)=0$,
          ήτοι το εκκρεμές ξεκινάει από τη θέση ηρεμίας χωρίς ταχύτητα.
          Η είσοδός μας είναι η ροπή
          \[
          u(t) \;=\; A_{0}\,\sin(\omega\,t)
          \]
          με τιμές $A_{0}=4$ και $\omega=2$. Αυτή προκαλεί περιοδική διέγερση
          στο εκκρεμές.

    \item \textbf{Αριθμητική Ολοκλήρωση:} \ 
          Η \en \texttt{ode45} \gr είναι ένας αριθμητικός αλγόριθμος τύπου \en Runge--Kutta \gr
          μεταβλητού βήματος, ο οποίος υπολογίζει ένα κατά προσέγγιση διάγραμμα
          $\bigl\{t_{k},\,x_{1}(t_{k}),\,x_{2}(t_{k})\bigr\}$ για
          $t\in[0,t_{\mathrm{final}}]$ (εδώ $t_{\mathrm{final}}=20\,\mathrm s$).
          Η παράμετρος \texttt{\en MaxStep=$10^{-3}$\gr} περιορίζει το μέγιστο
          βήμα ολοκλήρωσης σε $10^{-3}$, διασφαλίζοντας \emph{πυκνή} δειγματοληψία
          στον χρόνο και άρα μεγαλύτερη ακρίβεια στο αποτέλεσμα.

    \item \textbf{Αποτελέσματα:} \ 
          Αφού υπολογιστεί η λύση, μπορούμε να τη σχεδιάσουμε ως γραφικές 
          παραστάσεις $q(t)$ και $\dot{q}(t)$:
          \begin{itemize}
              \item Σε αρχικά χρονικά ($t$ κοντά στο μηδέν), το σύστημα βρίσκεται
                    σε μεταβατική φάση, αλλά σύντομα εξελίσσεται σε μια περιοδική
                    κίνηση που ακολουθεί τη συχνότητα της $\sin(\omega t)$.
              \item Η απόσβεση $c$ περιορίζει το εύρος ταλάντωσης, αποτρέποντας
                    ανεξέλεγκτη αύξηση των ταλαντώσεων.
              \item Για μεγάλα $t$, η κίνηση πλησιάζει μια περιοδική λύση σε συντονισμό
                    με την είσοδο $\omega=2$.
          \end{itemize}
\end{enumerate}

\hspace{-0.6cm}Με αυτόν τον τρόπο, \emph{χωρίς} να απαριθμήσουμε τον κώδικα, περιγράψαμε \emph{μαθηματικά}
το πώς καταλήγουμε να ορίσουμε το εκκρεμές ως σύστημα 1ης τάξης, να περάσουμε τις
συναρτήσεις του στο \en \texttt{ode45}\gr, και τελικά να πάρουμε το διάγραμμα
\[
\bigl(q(t),\;\dot{q}(t)\bigr), 
\quad
t\in[0,20].
\]

\noindent
Το \en \texttt{pendulum\_ode} \gr ορίζει τη διαφορική εξίσωση
\[
\dot{x}_1(t)=x_2(t), 
\quad
\dot{x}_2(t)=\frac{1}{mL^2}\Bigl[u(t)-c\,x_2(t)-m\,g\,L\,x_1(t)\Bigr],
\]
όπου $x_1(t)=q(t)$ και $x_2(t)=\dot{q}(t)$ όπως αναφέρθηκε και στο Θέμα~1(α).

\subsection*{Γραφικές Παράστασεις και Συμπεράσματα}

Το πρώτο διάγραμμα \en (Subplot~1) \gr δείχνει πώς μεταβάλλεται η \emph{γωνία} $q(t)$ στον χρόνο. 
Το δεύτερο διάγραμμα \en (Subplot~2) \gr δείχνει την \emph{γωνιακή ταχύτητα} $\dot{q}(t)$.

\vspace{0.5cm}
\hspace{-0.6cm}Παρακάτω παραθέτω και τα διαγράμματα:

\begin{figure}[H]
   \centering
   \includegraphics[width=1\textwidth]{latex_and_pdf/PendulumAngle.png}
   \caption{Παράδειγμα διαγράμματος εκκρεμούς.}
   \label{fig:myimage}
\end{figure}

\begin{itemize}
\item Παρατηρούμε ότι το σύστημα, λόγω της ημιτονοειδούς εισόδου $\sin(\omega t)$, 
παρουσιάζει μια μεταβατική φάση και στη συνέχεια κινείται περιοδικά.
\item Η απόσβεση $c=0.15$ επιδρά ώστε να περιοριστεί κάπως το εύρος ταλαντώσεων. 
\item Εάν τρέξουμε την προσομοίωση για περισσότερο χρόνο, θα δούμε ότι τελικά το 
σύστημα διαμορφώνει μία ημιτονοειδή έξοδο περίπου στην ίδια συχνότητα εισόδου 
(λόγω της γραμμικοποίησης).
\end{itemize}

\hspace{-0.6cm}Με βάση τα αποτελέσματα, επιβεβαιώνεται ότι η \en \texttt{ode45} \gr δίνει ένα \emph{ακριβές} 
διάγραμμα της λύσης στο πεδίο του χρόνου, επιτρέποντας να αναλύσουμε περαιτέρω την 
\emph{ευστάθεια}, το \emph{πλάτος ταλάντωσης}, κ.τ.λ.

\chapter{Εκτίμηση Παραμέτρων με Μέθοδο Ελαχίστων τετραγώνων}

\section{Θέμα 2(α) Γραμμική Παραμετροποίηση του Συστήματος}

\subsection*{Περιγραφή προβλήματος}
Θεωρούμε ότι το διάνυσμα κατάστασης \( x(t) = \begin{bmatrix} q(t) \\ \dot{q}(t) \end{bmatrix} \) και η είσοδος \( u(t) \) είναι μετρήσιμα μεγέθη. Στόχος μας είναι, χρησιμοποιώντας μετρήσεις των \( q(t), \dot{q}(t), u(t) \) με περίοδο δειγματοληψίας \( T_s = 0.1 \,\mathrm{sec} \), να εκτιμήσουμε τις φυσικές παραμέτρους του συστήματος \( m, L, c \) με τη μέθοδο των ελαχίστων τετραγώνων \en (Least Squares)\gr.
Χρησιμοποιείται σήμα εισόδου ημιτονοειδούς μορφής:  
\[
u(t) = A_0 \sin(\omega_0 t)
\]

\subsection*{Μοντελοποίηση}
Το σύστημα περιγράφεται από την εξίσωση:

\begin{equation}
\ddot{q} = -\frac{c}{mL^2} \dot{q} - \frac{g}{L} q + \frac{1}{mL^2} u
\end{equation}

\hspace{-0.6cm}Θέτουμε:

\[
a_1 = \frac{c}{mL^2}, \quad a_2 = \frac{g}{L}, \quad b_0 = \frac{1}{mL^2}
\]

\hspace{-0.6cm}και επομένως:

\begin{equation}
\ddot{q} = -a_1 \dot{q} - a_2 q + b_0 u
\end{equation}

\hspace{-0.6cm}Από την παραπάνω μορφή, ορίζουμε:

\[
\theta^T = [a_1 \quad a_2 \quad b_0], \quad \phi = [-\dot{q} \quad -q \quad u]^T
\]

\hspace{-0.6cm}και επομένως:

\begin{equation}
\ddot{q} = \theta^T \phi
\end{equation}

\hspace{-0.6cm}Ωστόσο, η επιτάχυνση \( \ddot{q} \) δεν είναι άμεσα μετρήσιμη, οπότε εφαρμόζουμε γραμμική παραμετροποίηση και χρήση φίλτρου ώστε να καταλήξουμε σε ισοδύναμη μορφή μόνο με μετρήσιμες ποσότητες.

\subsection*{Μορφή Φιλτραρισμένου Μοντέλου (Εφαρμογή Φίλτρου 1ης Τάξης)}

Για φίλτρα ανώτερης τάξης, όπως \(\Lambda(s) = s^n + \lambda_1 s^{n-1} + \dots + \lambda_n\), η εξίσωση που περιγράφει την έξοδο του φιλτραρίσματος της παραγώγου γίνεται:

\[
z = \frac{s}{\Lambda(s)} \dot{q} = \frac{\Lambda(s) - \lambda^\top \Delta_{n-2}(s)}{\Lambda(s)} \dot{q}
\]

\[
\Rightarrow \dot{q} = \dot{y} - \lambda^\top \cdot \frac{\Delta_{n-2}(s)}{\Lambda(s)} \dot{q}
\]

\hspace{-0.6cm}όπου:
\(\Delta_{n-2}(s)\) είναι το πολυώνυμο βαθμού \(n-2\),
\(\lambda^\top\) το διάνυσμα συντελεστών του φίλτρου,
και η εξίσωση προκύπτει από την επέκταση της ταυτότητας \(\frac{s}{\Lambda(s)} = 1 - \frac{\lambda}{\Lambda(s)}\), για γενική \(\Lambda(s)\).

\hspace{-0.6cm}Αυτό επιτρέπει τη φιλτραρισμένη εκτίμηση της παραγώγου χωρίς να απαιτείται πραγματικός διαφορικός τελεστής, κάτι που είναι κρίσιμο σε διακριτό χρόνο και σε δεδομένα με θόρυβο.

\hspace{-0.6cm}Δεδομένου ότι δεν έχουμε άμεση πρόσβαση στο \( \ddot{q} \), εφαρμόζουμε ευσταθές φίλτρο πρώτης τάξης στο σύστημα ώστε να γράψουμε την εξίσωση σε μορφή μετρήσιμων μεταβλητών. Δηλαδή:

\begin{equation}
z = \frac{s}{\Lambda(s)} \dot{q} = \frac{\Lambda(s) - \lambda}{\Lambda(s)} \dot{q}
\Rightarrow 
\dot{q} = z + \lambda \cdot \frac{1}{\Lambda(s)} \dot{q}
\end{equation}

\hspace{-0.6cm}Η έξοδος της διαδικασίας δίνεται από:

\begin{equation}
\dot{q} = \theta_1 \zeta_1 + \theta_2 \zeta_2 + \theta_3 \zeta_3 - \lambda \zeta_1
\Rightarrow
\dot{q} = (\theta_1 - \lambda) \zeta_1 + \theta_2 \zeta_2 + \theta_3 \zeta_3
\end{equation}

\hspace{-0.6cm}όπου:

\begin{equation}
\zeta = 
\left[
\frac{-1}{\Lambda(s)} \dot{q} \quad
\frac{-1}{\Lambda(s)} q \quad
\frac{1}{\Lambda(s)} u
\right]^T
\end{equation}

\hspace{-0.6cm}δηλαδή συνοπτικά:
\[
\dot{q} = \theta^\top
\left[
\begin{array}{c}
-\frac{1}{\Lambda(s)} \dot{q} \\
-\frac{1}{\Lambda(s)} q \\
\frac{1}{\Lambda(s)} u
\end{array}
\right]
+ \lambda \cdot \frac{1}{\Lambda(s)} \dot{q}
\]

\subsection*{Εκτίμηση με \en Least Squares \gr}

Ορίζουμε:

\[
\Phi = 
\begin{bmatrix}
\zeta_1 & \zeta_2 & \zeta_3
\end{bmatrix},
\quad
Y = \dot{q}
\]

\hspace{-0.6cm}και εφαρμόζουμε την κλασική εξίσωση ελαχίστων τετραγώνων:

\begin{equation}
\hat{\boldsymbol{\theta}} = (\Phi^T \Phi)^{-1} \Phi^T Y
\end{equation}

\hspace{-0.6cm}Από την εκτίμηση \( \hat{\boldsymbol{\theta}} = [\hat{a}_1, \hat{a}_2, \hat{b}_0]^T \), ανακτώνται οι φυσικές παράμετροι του συστήματος:

\[
\hat{L} = \frac{g}{\hat{a}_2}, \quad
\hat{m} = \frac{1}{\hat{b}_0 \hat{L}^2}, \quad
\hat{c} = (\hat{a}_1 + \lambda) \hat{m} \hat{L}^2
\]

\subsection*{Υλοποίηση Φίλτρου}

Για την εκδοχή του φίλτρου \( \frac{1}{s + \lambda} \), εφαρμόζεται η προσέγγιση \en Tustin \gr, η οποία δίνει την εξής διακριτή υλοποίηση:

\begin{equation}
y_f[k] = \alpha \cdot y_f[k-1] + \beta \cdot (x[k] + x[k-1])
\end{equation}

\hspace{-0.6cm}όπου:

\[
\alpha = \frac{2 - T_s \lambda}{2 + T_s \lambda}, \quad
\beta = \frac{T_s}{2 + T_s \lambda}
\]

\hspace{-0.6cm}και \( T_s \) το διάστημα δειγματοληψίας.

\subsection*{Αποτελέσματα}

Αφού εφαρμόσουμε την παραπάνω διαδικασία στο σύστημα, λαμβάνουμε τις ακόλουθες εκτιμήσεις:

\begin{itemize}
    \item \textbf{Πραγματικές Παράμετροι:}
    \en
    \begin{itemize}
        \item \( L = 1.2500 \) m
        \item \( m = 0.7500 \) kg
        \item \( c = 0.1500 \) N·m·s
    \end{itemize}
    \gr
    \item \textbf{Εκτιμώμενες Παράμετροι:}
    \en
    \begin{itemize}
        \item \( \hat{L} = 1.2416 \) m
        \item \( \hat{m} = 0.7525 \) kg
        \item \( \hat{c} = 0.1502 \) N·m·s
    \end{itemize}
    \gr
    \item \textbf{\en MSE \gr Σφάλμα Προσομοίωσης:}
    \en
    \begin{itemize}
        \item \( \text{MSE}_q = 0.000401 \)
        \item \( \text{MSE}_{\dot{q}} = 0.003194 \)
    \end{itemize}
    \gr
\end{itemize}
Παρακάτω παρατίθεται και το διάγραμμα για οπτικοποίηση των σφαλμάτων αλλά και των αποκρίσεων \( q(t) \), \( \dot{q}(t) \), καθώς και των αντίστοιχων εκτιμήσεών τους \( \hat{q}(t) \), \( \hat{\dot{q}}(t) \). Παρατηρούμε ότι τα σφάλματα παραμένουν μικρά καθ’ όλη τη διάρκεια της προσομοίωσης, γεγονός που επαληθεύει την ορθότητα της εκτίμησης μέσω φίλτρου πρώτης τάξης και μεθόδου ελαχίστων τετραγώνων.

\begin{figure}[H]
    \centering
    \includegraphics[width=1.0\textwidth]{latex_and_pdf/2_a_pic.png}
    \caption{Αποτελέσματα Θέματος 2(α): \(q(t)\), \(\dot{q}(t)\), και τα αντίστοιχα σφάλματα}
\end{figure}


\section{Θέμα 2(β) Μετρήσιμα γωνία \en q(t) \gr και είσοδος \en u(t) \gr}

Στο \emph{Θέμα~2(β)} επιθυμούμε να εκτιμήσουμε τις φυσικές παραμέτρους του 
(γραμμικοποιημένου) εκκρεμούς -- δηλαδή $m$, $L$ και $c$ -- \emph{χρησιμοποιώντας 
μόνο την έξοδο} $\bigl\{q(t)\bigr\}$ και την είσοδο $\bigl\{u(t)\bigr\}$, \emph{χωρίς} 
άμεση μέτρηση της $\dot{q}(t)$. 

%%%%%%%%%%%%%%%%%%%%%%%%%%%%%%%%%%%%%%%%%%%%%%%%%%%%%%%%%%%%
\subsection*{Βασική Ιδέα}

Για να αποφύγουμε την αριθμητική \emph{παραγωγό} του $q(t)$ (που συχνά είναι 
ευαίσθητη σε θόρυβο), εφαρμόζουμε ένα \emph{σταθερό φίλτρο} 
$\tfrac{1}{\Lambda(s)} = \tfrac{1}{s^2+3s+2}$ τόσο στην είσοδο $u(t)$ 
όσο και στην έξοδο $q(t)$. Έτσι οι άγνωστες παράγωγοι $\dot{q}(t),\;\ddot{q}(t)$ 
αντικαθίστανται από ολοκληρώσεις (ή ισοδύναμα, διαφορικές εξισώσεις) στο πεδίο του χρόνου. 

\vspace{0.4cm}

\hspace{-0.6cm}Ξεκινάμε από τη διαφορική εξίσωση:
\[
\ddot{q}(t) + a_{1}\,\dot{q}(t) + a_{2}\,q(t)
\;=\;
b_{0}\,u(t),
\]
όπου
\[
a_{1} \;=\;\frac{c}{m\,L^{2}},
\quad
a_{2} \;=\;\frac{g}{L},
\quad
b_{0} \;=\;\frac{1}{m\,L^{2}}.
\]
Για να αποφύγουμε τη μέτρηση (ή αριθμητική παραγώγιση) των \(\dot{q},\ddot{q}\),
πολλαπλασιάζουμε και τις δύο πλευρές της εξίσωσης με το σταθερό φίλτρο
\(\tfrac{1}{\Lambda(s)}\), όπου
\[
\Lambda(s) \;=\; s^2 + 3\,s + 2.
\]
Ορίζουμε τα \emph{φιλτραρισμένα} σήματα:
\[
y_{2}(t) \;=\;\frac{s^2}{\Lambda(s)}\,q(t),
\quad
y_{1}(t) \;=\;\frac{s}{\Lambda(s)}\,q(t),
\quad
y_{0}(t) \;=\;\frac{1}{\Lambda(s)}\,q(t),
\quad
u_{0}(t) \;=\;\frac{1}{\Lambda(s)}\,u(t).
\]
Τότε, η εξίσωση γράφεται:
\[
y_{2}(t)
\;=\;
a_{1}\,\bigl[-\,y_{1}(t)\bigr]
\;+\;
a_{2}\,\bigl[-\,y_{0}(t)\bigr]
\;+\;
b_{0}\,u_{0}(t),
\]
το οποίο είναι \emph{γραμμικό} ως προς \(\theta = [\,a_{1},\,a_{2},\,b_{0}]^T\).
Συνεπώς, αν ορίσουμε το διάνυσμα
\[
\zeta(t) \;=\;
\begin{bmatrix}
-\dfrac{s}{\Lambda(s)}\,q(t)
\\[6pt]
-\dfrac{1}{\Lambda(s)}\,q(t)
\\[6pt]
\dfrac{1}{\Lambda(s)}\,u(t)
\end{bmatrix}
\;=\;
\begin{bmatrix}
-\,y_{1}(t)
\\[3pt]
-\,y_{0}(t)
\\[3pt]
u_{0}(t)
\end{bmatrix},
\]
τότε η σχέση συνοψίζεται ως
\[
y_{2}(t) \;=\;\theta^{T}\,\zeta(t).
\]
Με δειγματοληψία στο χρονικό διάστημα \([0,\,10]\) (π.χ.\ με βήμα
\(T_{s}=0.1\)) σχηματίζουμε μία \emph{γραμμική παλινδρόμηση} και λύνουμε το
πρόβλημα ελαχίστων τετραγώνων:
\[
\hat{\theta}
\;=\;
\bigl(\Phi^{T}\,\Phi\bigr)^{-1}\,\Phi^{T}\,\mathbf{y}_{2},
\]
όπου ο πίνακας \(\Phi\) έχει ως κάθε γραμμή το διάνυσμα \(\zeta(t_k)^T\) και το διάνυσμα \(\mathbf{y}_2\) αποτελείται από τις τιμές \(y_2(t_k)\).

\paragraph{Αποτελέσματα:}
Με τις πραγματικές τιμές
\[
m_{\mathrm{true}}=0.75,\quad
L_{\mathrm{true}}=1.25,\quad
c_{\mathrm{true}}=0.15,
\]
πήραμε τις παρακάτω \emph{εκτιμήσεις}:
\en
\[
\begin{aligned}
a_{1} \;=\;\tfrac{c}{mL^{2}}&:\quad 0.12800 \;\;\text{(true)} \quad\rightarrow\quad 0.12330\;(\text{est}),\\
a_{2} \;=\;\tfrac{g}{L}&:\quad 7.84800 \;\;\text{(true)} \quad\rightarrow\quad 7.81378\;(\text{est}),\\
b_{0} \;=\;\tfrac{1}{mL^{2}}&:\quad 0.85333 \;\;\text{(true)} \quad\rightarrow\quad 0.84599\;(\text{est}).
\end{aligned}
\]
\gr
Από αυτά, ανακτήθηκαν οι φυσικές παράμετροι:
\[
\begin{aligned}
m&:\quad 0.75000\;\;\rightarrow\;0.74993,\\
L&:\quad 1.25000\;\;\rightarrow\;1.25547,\\
c&:\quad 0.15000\;\;\rightarrow\;0.14574.
\end{aligned}
\]
Βλέπουμε ότι οι εκτιμήσεις πλησιάζουν πολύ κοντά στις πραγματικές τιμές,
επιβεβαιώνοντας την εγκυρότητα της φιλτραρισμένης μεθόδου παραμετρικής εκτίμησης
\en (Filtered Regression) \gr για το πρόβλημα του εκκρεμούς.

\vspace{0.5cm}

\smallskip

\noindent
\textbf{Συμπέρασμα:} Η διαδικασία αυτή, γνωστή και ως \en \emph{``filtered regression''}, \gr
είναι ένα τυπικό παράδειγμα πώς μπορεί κανείς να εκτιμήσει παραμέτρους σε \emph{δεύτερης 
τάξης συστήματα} χωρίς να μετρά τις παράγωγους της εξόδου. Αρκεί να πραγματοποιηθεί 
πολλαπλασιασμός (ή εφαρμογή) ενός σταθερού φίλτρου $\tfrac{1}{\Lambda(s)}$, 
ώστε η διαφορική εξίσωση να μετατραπεί σε \emph{γραμμική σχέση} ως προς τις αγνώστους 
$a_1,a_2,b_0$. 

\vspace{0.5cm}

\hspace{-0.6cm}Παρακάτω παρατίθενται τα διαγράμματα που οπτικοποιούν τα αποτελέσματα της προσομοίωσης του συστήματος του εκκρεμούς, όπου εφαρμόζεται η μέθοδος εκτίμησης παραμέτρων με βάση την τεχνική των ελαχίστων τετραγώνων και τη φιλτραρισμένη παλινδρόμηση. Συγκεκριμένα, τα διαγράμματα που θα παρουσιαστούν περιλαμβάνουν:

\begin{itemize}
    \item \textbf{Διάγραμμα γωνίας \(q(t)\) ως προς το χρόνο:} Σε αυτό, απεικονίζεται η πραγματική εξέλιξη της γωνίας του εκκρεμούς και η αντίστοιχη απόκριση που λαμβάνεται με τις εκτιμημένες παραμέτρους. Η σύγκριση αυτή μας επιτρέπει να αξιολογήσουμε την ακρίβεια της παραμετροποίησης.
    
    \item \textbf{Διάγραμμα γωνιακής ταχύτητας \(\dot{q}(t)\) ως προς το χρόνο:} Μέσω αυτού του διαγράμματος μπορούμε να παρακολουθήσουμε τη δυναμική συμπεριφορά του συστήματος, επαληθεύοντας τη συνεκτικότητα μεταξύ της πραγματικής κίνησης και της προβλεπόμενης από το μοντέλο.
    
    \item \textbf{Διάγραμμα σφάλματος \(e_q(t)=q(t)-\hat{q}(t)\):} Αυτό το διάγραμμα αναδεικνύει τις διαφορές μεταξύ της πραγματικής εξόδου και της απόκρισης που προβλέπει το μοντέλο με τις εκτιμημένες παραμέτρους. Μια μικρή τιμή του σφάλματος υποδεικνύει ότι το μοντέλο αναπαριστά επαρκώς τη δυναμική του εκκρεμούς.
\end{itemize}

\begin{figure}[H]
   \centering
   \includegraphics[width=1\textwidth]{latex_and_pdf/ErrorAndAngles.png}
   \caption{Παράδειγμα διαγράμματος εκκρεμούς.}
   \label{fig:myimage}
\end{figure}

\chapter{\gr Μεταβολή παραμέτρων και επίδραση θορύβου}
\section{Θέμα 3(α) Προσθήκη Λευκού Γκαουσιανού Θορύβου στα Δεδομένα}

Σε αυτήν την ενότητα, μελετάμε πώς η παρουσία θορύβου στα μετρούμενα δεδομένα $\{q(t_k)\}$ 
επηρεάζει την ακρίβεια της εκτίμησης των παραμέτρων του συστήματος, 
χρησιμοποιώντας την ίδια φιλτραρισμένη μέθοδο από το Θέμα~2(β) η οποία αξίζει να σημειώσω ότι χρησιμοποιήθηκε και στα ερωτήματα (β) και (γ) αντίστοιχα του Θέματος 3.

\subsection*{Σύνοψη της Διαδικασίας}

\begin{enumerate}
    \item \textbf{Αρχική προσομοίωση \en (``noiseless'') \gr}: 
    Όπως και στο Θέμα~2(β), ξεκινάμε προσομοιώνοντας το ``αληθινό'' σύστημα με τις 
    πραγματικές τιμές $m_{\mathrm{true}}, L_{\mathrm{true}}, c_{\mathrm{true}}$. 
    Δειγματοληπτούμε τα σήματα $q(t)$ και $u(t)$ σε βήμα $T_s=0.1\,\mathrm s$ 
    και εφαρμόζουμε τη φιλτραρισμένη παλινδρόμηση, καταλήγοντας σε εκτιμήσεις 
    $\{\hat{a}_1,\hat{a}_2,\hat{b}_0\}$ και, συνακόλουθα, 
    $\{\hat{m},\hat{L},\hat{c}\}$.

    \item \textbf{Προσθήκη λευκού γκαουσιανού θορύβου}: 
    Για να προσομοιώσουμε την παρουσία θορυβώδους μέτρησης, 
    \emph{προσθέτουμε} έναν λευκό γκαουσιανό θόρυβο στο δειγματοληπτημένο σήμα $q(t_k)$:
    \[
    q_{\mathrm{noisy}}(t_k) \;=\; q(t_k) \;+\; \eta(t_k),
    \]
    όπου $\eta(t_k)\sim \mathcal{N}(0,\,\sigma^2)$, με $\sigma$ μια σταθερά που 
    καθορίζει το επίπεδο θορύβου \en (\texttt{noise\_std = 0.04})\gr.

    \item \textbf{Εκτίμηση παραμέτρων με θορυβώδη δεδομένα}:
    Επαναλαμβάνουμε \emph{ακριβώς} τα βήματα της φιλτραρισμένης μεθόδου από το Θέμα~2(β) 
    (πολλαπλασιασμός με $\tfrac{1}{s^2+3s+2}$, διαμόρφωση γραμμικής παλινδρόμησης, 
    λύση ελαχίστων τετραγώνων), μόνο που τώρα τα δείγματα $q(t_k)$ έχουν θόρυβο. 
    Αποκτούμε έτσι \(\hat{a}_1^{(\mathrm{noisy})},\,\hat{a}_2^{(\mathrm{noisy})},\,\hat{b}_0^{(\mathrm{noisy})}\) 
    και εξάγουμε τις αντίστοιχες εκτιμήσεις $\{\hat{m},\hat{L},\hat{c}\}$.

    \item \textbf{Σύγκριση αποτελεσμάτων}:
    \begin{itemize}
        \item \emph{Ποσοτική σύγκριση}: Ελέγχουμε κατά πόσο οι εκτιμήσεις 
        $\hat{m}_{\mathrm{noisy}},\,\hat{L}_{\mathrm{noisy}},\,\hat{c}_{\mathrm{noisy}}$ 
        διαφέρουν από τις πραγματικές τιμές ή από τις τιμές που εκτιμήθηκαν 
        \emph{χωρίς} θόρυβο. 
        \item \emph{Ποιοτική σύγκριση}: Προσομοιώνουμε το σύστημα με τις 
        ``θορυβώδεις'' εκτιμήσεις και σχεδιάζουμε το σφάλμα 
        \(\displaystyle e_q(t)=q(t)-\hat{q}(t)\). Εάν ο θόρυβος δεν είναι υπερβολικά μεγάλος, 
        το σφάλμα παραμένει σχετικά μικρό.
    \end{itemize}
\end{enumerate}

\subsection*{Βασικές Παρατηρήσεις}

\begin{itemize}
    \item \textbf{Ευαισθησία στη στάθμη θορύβου}: 
    Όσο μεγαλώνει η τιμή \(\sigma\), τόσο αυξάνεται η απόκλιση των 
    $\hat{m},\hat{L},\hat{c}$ από τις πραγματικές τιμές, λόγω της αριθμητικής αστάθειας 
    που εμφανίζεται στη φιλτραρισμένη παλινδρόμηση.

    \item \textbf{Ο ρόλος του φίλτρου}:
    Η χρήση του φίλτρου \(\tfrac{1}{s^2+3s+2}\) \emph{αποτρέπει} την άμεση 
    διαφοροποίηση των θορυβωδών δεδομένων, η οποία θα ήταν πιο ευαίσθητη στο θόρυβο. 
    Παρόλα αυτά, η μέθοδος εξακολουθεί να επηρεάζεται από μεγάλο επίπεδο θορύβου, 
    καθώς το φιλτράρισμα δεν εξαλείφει πλήρως τις αβεβαιότητες.

    \item \textbf{Πρακτικό όφελος}:
    Ακόμα και με την προσθήκη θορύβου, οι εκτιμήσεις συνήθως διατηρούνται 
    αρκετά \emph{κοντά} στις πραγματικές τιμές, επιτρέποντας μία ρεαλιστική 
    αναπαράσταση της δυναμικής του εκκρεμούς.
\end{itemize}

\noindent
Συνολικά, το πείραμα του Θέματος~3(α) καταδεικνύει ότι η μέθοδος \emph{ελαχίστων τετραγώνων 
με φιλτράρισμα} παραμένει σχετικά \emph{ανθεκτική} σε μέτριες στάθμες λευκού 
γκαουσιανού θορύβου, παρέχοντας ικανοποιητικές εκτιμήσεις των παραμέτρων του εκκρεμούς.

\vspace{0.3cm}

\hspace{-0.6cm}Τα αποτελέσματα της βασικής (χωρίς θόρυβο) μεθόδου ήταν:
\[
\begin{array}{ll}
a_{1} = \dfrac{c}{mL^2}: & \quad \text{πραγματική: } 0.12800 \quad \text{\en vs\gr} \quad \text{εκτιμημένη: } 0.12330, \\[2ex]
a_{2} = \dfrac{g}{L}:   & \quad \text{πραγματική: } 7.84800 \quad \text{\en vs\gr} \quad \text{εκτιμημένη: } 7.81378, \\[2ex]
b_{0} = \dfrac{1}{mL^2}:& \quad \text{πραγματική: } 0.85333 \quad \text{\en vs\gr} \quad \text{εκτιμημένη: } 0.84599, \\[2ex]
m:                     & \quad \text{πραγματική: } 0.75000 \quad \text{\en vs
\gr} \quad \text{εκτιμημένη: } 0.74993, \\[2ex]
L:                     & \quad \text{πραγματική: } 1.25000 \quad \text{\en vs
\gr} \quad \text{εκτιμημένη: } 1.25547, \\[2ex]
c:                     & \quad \text{πραγματική: } 0.15000 \quad \text{\en vs\gr} \quad \text{εκτιμημένη: } 0.14574.
\end{array}
\]

\vspace{0.3cm}

\hspace{-0.6cm}Μετά την προσθήκη του θορύβου, τα αποτελέσματα της εκτίμησης έγιναν:
\[
\begin{array}{ll}
a_{1}: & \quad 0.16389, \\[2ex]
a_{2}: & \quad 8.03465, \\[2ex]
b_{0}: & \quad 0.90568, \\[2ex]
m:     & \quad 0.74066, \\[2ex]
L:     & \quad 1.22096, \\[2ex]
c:     & \quad 0.18096.
\end{array}
\]

\subsection*{Ανάλυση Αποτελεσμάτων}

Παρατηρούμε τα εξής:
\begin{itemize}
    \item Η εκτίμηση του \(a_{1}\) αυξάνεται από 0.12330 σε 0.16389, πράγμα που υποδηλώνει υπερεκτίμηση του συντελεστή απόσβεσης όταν εισάγεται θόρυβος.
    \item Το \(a_{2}\) και το \(b_{0}\) επίσης παρουσιάζουν αύξηση, με αποτέλεσμα οι υπολογισμένες τιμές του μήκους και της μάζας να αποκλίνουν: το εκτιμημένο \(L\) μειώνεται (από 1.25547 σε 1.22096) και το εκτιμημένο \(m\) μειώνεται ελαφρώς (από 0.74993 σε 0.74066).
    \item Τέλος, ο υπολογισμός του \(c\) (που δίνεται από \(c = a_1 \,m\,L^2\)) αυξάνεται, φανερώνοντας μια συνολική μεταβολή της δυναμικής συμπεριφοράς του συστήματος λόγω του θορύβου.
\end{itemize}

\subsection*{Παρουσίαση Διαγράμματος}

Στο Διάγραμμα~\ref{fig:plot3a} παρουσιάζονται τα διαγράμματα του σφάλματος
\(e_q(t)=q(t)-\hat{q}(t)\) για την περίπτωση χωρίς θόρυβο και για την περίπτωση με θόρυβο (με \(\sigma=0.04\)). 

\begin{figure}[H]
    \centering
    \includegraphics[width=1\textwidth]{plot3a.png}
    \caption{Σύγκριση του σφάλματος \(e_q(t)\) μεταξύ της περίπτωσης χωρίς θόρυβο (π.χ., μαύρη) και της περίπτωσης με θόρυβο (π.χ., πράσινη).}
    \label{fig:plot3a}
\end{figure}

Η ανάλυση των διαγραμμάτων αποδεικνύει ότι:
\begin{itemize}
    \item Το σφάλμα για την περίπτωση χωρίς θόρυβο είναι ελάχιστο, επιβεβαιώνοντας την ακρίβεια της μεθόδου όταν τα δεδομένα είναι καθαρά.
    \item Η εισαγωγή θορύβου οδηγεί σε αύξηση του σφάλματος, κάτι που είναι αναμενόμενο λόγω της ευαισθησίας της μεθόδου στα θορυβώδη δεδομένα.
\end{itemize}

\noindent
\textbf{Συμπέρασμα:} Η μέθοδος φιλτραρισμένης παλινδρόμησης, παρά το γεγονός ότι παρουσιάζει κάποια ευαισθησία στα θορυβώδη δεδομένα, εξακολουθεί να παρέχει αρκετά καλή προσέγγιση των παραμέτρων του συστήματος, καθώς και την αντίστοιχη επαναπροσομοίωση του συστήματος. Ωστόσο, η παρουσία θορύβου επιδεινώνει το σφάλμα εκτίμησης και πρέπει να λαμβάνονται μέτρα για την αντιμετώπισή του σε εφαρμογές υψηλής ακρίβειας.

\section{Θέμα 3(β): Μελέτη της Επίδρασης της Περιόδου Δειγματοληψίας \(T_s\)}

Στο δεύτερο ερώτημα του Θέματος~3(β), εξετάζουμε πώς η \emph{μεταβολή} της περιόδου δειγματοληψίας \(T_s\) επηρεάζει την ακρίβεια των εκτιμήσεων των παραμέτρων \(\{m,\,L,\,c\}\). Η διαδικασία ακολουθεί τα ίδια βήματα φιλτραρισμένης παλινδρόμησης, όπως παρουσιάστηκαν προηγουμένως, αλλά επαναλαμβάνεται για διαφορετικές τιμές \(T_s\) (π.χ.\ \(0.01,\,0.05,\,0.1,\,0.2,\,0.5\)\,\en s\gr).

\subsection*{Διαδικασία}

\begin{enumerate}
    \item \textbf{Προσομοίωση ``αληθινού'' συστήματος:} 
    Όπως πάντα, παράγουμε τα σήματα \(\{q(t),u(t)\}\) με βάση τις πραγματικές τιμές \(m_{\mathrm{true}},\,L_{\mathrm{true}},\,c_{\mathrm{true}}\) και τη συνάρτηση \(\sin(\omega t)\) για την είσοδο.
    \item \textbf{Επαναλαμβανόμενη Δειγματοληψία:} 
    Για κάθε διαφορετική τιμή του \(T_s\), δημιουργούμε ένα νέο σύνολο δειγμάτων \(\bigl\{t_k,\,q(t_k),\,u(t_k)\bigr\}\). 
    \item \textbf{Φιλτραρισμένη Παλινδρόμηση:} 
    Εφαρμόζουμε το ίδιο φίλτρο \(\tfrac{1}{s^2+3s+2}\) και επιλύουμε το πρόβλημα ελαχίστων τετραγώνων ώστε να εκτιμήσουμε \(\{\hat{a}_1,\hat{a}_2,\hat{b}_0\}\) και συνακόλουθα \(\{\hat{m},\hat{L},\hat{c}\}\).
    \item \textbf{Καταγραφή Σφαλμάτων:} 
    Αποθηκεύουμε την απόκλιση των εκτιμήσεων \(\hat{m},\hat{L},\hat{c}\) από τις πραγματικές τιμές σε συνάρτηση με το \(T_s\).
\end{enumerate}

\subsection*{Αποτελέσματα}

Τα αποτελέσματα απεικονίζονται γραφικά στο Διάγραμμα~\ref{fig:tsVary}, όπου παρουσιάζονται οι διαφορές \en (errors) \gr \(\Delta m = \hat{m}-m_{\mathrm{true}}\), \(\Delta L = \hat{L}-L_{\mathrm{true}}\) και \(\Delta c = \hat{c}-c_{\mathrm{true}}\) σε σχέση με την περίοδο δειγματοληψίας \(T_s\).

\begin{figure}[H]
    \centering
    \includegraphics[width=1.0\textwidth]{latex_and_pdf/ts_vary.png}
    \caption{Επίδραση της μεταβολής του \(T_s\) στα σφάλματα εκτίμησης των παραμέτρων \(\hat{a_1},\hat{a_2},\hat{b_0}\).}
    \label{fig:tsVary}
\end{figure}

\begin{figure}[H]
    \centering
    \includegraphics[width=1.0\textwidth]{latex_and_pdf/errorMLC.png}
    \caption{Επίδραση της μεταβολής του \(T_s\) στα σφάλματα εκτίμησης των \(\hat{m},\hat{L},\hat{c}\).}
    \label{fig:tsVary}
\end{figure}

\noindent
Όπως διαπιστώνεται:
\begin{itemize}
    \item \textbf{Για μικρό \(T_s\)} (π.χ.\ \(0.01\)\,\en s\gr), οι εκτιμήσεις είναι ιδιαίτερα ακριβείς, καθώς η συχνή δειγματοληψία παρέχει πλούσιες πληροφορίες στο φίλτρο και στη μέθοδο ελαχίστων τετραγώνων.
    \item \textbf{Για μεγάλο \(T_s\)} (π.χ.\ \(0.5\)\,\en s \gr έως και \(1.0\)\,\en s\gr), τα σφάλματα αυξάνονται αισθητά, διότι η σπάνια δειγματοληψία οδηγεί σε φτωχότερη καταγραφή της δυναμικής του συστήματος.
    \item \textbf{Συμβιβασμός ταχύτητας/ακρίβειας}: Ενώ η πολύ μικρή περίοδος δειγματοληψίας βελτιώνει την ακρίβεια, αυξάνει το υπολογιστικό κόστος. Στην πράξη, επιλέγεται μία ενδιάμεση τιμή \(T_s\) που επιτυγχάνει επαρκή ακρίβεια χωρίς υπέρμετρο υπολογιστικό φόρτο.
\end{itemize}

\subsection*{Συμπέρασμα}

Το πείραμα του Θέματος~3(β) καταδεικνύει ότι η μέθοδος φιλτραρισμένης παλινδρόμησης εξαρτάται ουσιωδώς από την ποιότητα και την πυκνότητα των δεδομένων που παρέχει η δειγματοληψία. Μια \emph{πολύ αραιή} δειγματοληψία μπορεί να οδηγήσει σε σημαντικά σφάλματα εκτίμησης, ενώ η \emph{πιο συχνή} δειγματοληψία βελτιώνει την ακρίβεια αλλά αυξάνει το υπολογιστικό κόστος. Ωστόσο, για ένα ευρύ φάσμα ενδιάμεσων τιμών \(T_s\), η μέθοδος παραμένει αρκετά αξιόπιστη, δίνοντας εκτιμήσεις παραμέτρων πολύ κοντά στις πραγματικές τιμές του συστήματος.

\section{Θέμα 3(γ): Μελέτη της Επίδρασης του Πλάτους \(A_0\)}

Στο τρίτο ερώτημα του Θέματος~3(γ), εξετάζουμε πώς η \emph{μεταβολή} του πλάτους \(A_0\) της ημιτονοειδούς εισόδου \(u(t)=A_0\sin(\omega t)\) επηρεάζει την ακρίβεια των εκτιμήσεων των παραμέτρων \(\{m,L,c\}\). Ακολουθείται η ίδια \emph{φιλτραρισμένη} μέθοδος εκτίμησης, όπως περιγράφηκε στα προηγούμενα μέρη, αλλά επαναλαμβάνεται η διαδικασία για διάφορες τιμές \(A_0\) (π.χ.\ \(1,2,3,4,5,8\)).

\subsection*{Διαδικασία}

\begin{enumerate}
    \item \textbf{Προσομοίωση ``αληθινού'' συστήματος:}
    Χρησιμοποιούμε τις πραγματικές τιμές \(m_{\mathrm{true}},L_{\mathrm{true}},c_{\mathrm{true}}\) και μια βασική γωνιακή συχνότητα \(\omega=2\). Προσομοιώνουμε το σύστημα με $u(t)=A_0 \sin(\omega t)$ για ένα εύρος \(A_0\).
    
    \item \textbf{Δειγματοληψία:}
    Για κάθε τιμή του \(A_0\), λαμβάνουμε δείγματα των σημάτων \(\bigl\{q(t_k),\,u(t_k)\bigr\}\) με περίοδο \(T_s=0.1\,\mathrm s\).

    \item \textbf{Φιλτραρισμένη Παλινδρόμηση:}
    Εφαρμόζουμε το φίλτρο \(\tfrac{1}{s^2 + 3s + 2}\) στα δείγματα, σχηματίζουμε τη γραμμική σχέση 
    \[
    y_2(t) = a_1[-y_1(t)] + a_2[-y_0(t)] + b_0\,u_0(t),
    \]
    και λύνουμε σε ελάχιστα τετράγωνα για να εκτιμήσουμε \(\hat{a}_1,\hat{a}_2,\hat{b}_0\). Στη συνέχεια, ανακτούμε \(\hat{m},\hat{L},\hat{c}\).

    \item \textbf{Συλλογή σφαλμάτων:}
    Καταγράφουμε τις διαφορές \(\Delta a_1 = \hat{a}_1 - a_1,\;\Delta a_2 = \hat{a}_2 - a_2,\;\Delta b_0 = \hat{b}_0 - b_0\) σε συνάρτηση με το \(A_0\). Προαιρετικά, μπορούμε να υπολογίσουμε και τις αποκλίσεις \(\Delta m,\,\Delta L,\,\Delta c\).
\end{enumerate}

\subsection*{Αποτελέσματα}

Τα αποτελέσματα απεικονίζονται στο Διάγραμμα~\ref{fig:ampVaryMLC}, όπου φαίνονται τα σφάλματα εκτίμησης \(\Delta m,\Delta L,\Delta c\) ως συνάρτηση του πλάτους \(A_0\).

\begin{figure}[H]
    \centering
    \includegraphics[width=1\textwidth]{A0_vary.png}
    \caption{3(γ): Σφάλματα εκτίμησης των παραμέτρων $a_1,a_2,b_0$ σε συνάρτηση με το πλάτος εισόδου $A_0$.}
    \label{fig:ampVary}
\end{figure}

\begin{figure}[H]
    \centering
    \includegraphics[width=1\textwidth]{latex_and_pdf/errorMLCA0.png}
    \caption{3(γ): Σφάλματα εκτίμησης των παραμέτρων $m,L,c$ σε συνάρτηση με το πλάτος εισόδου $A_0$.}
    \label{fig:ampVaryMLC}
\end{figure}


\noindent
Από την ανάλυση των διαγραμμάτων προκύπτουν τα εξής συμπεράσματα:
\begin{itemize}
    \item \textbf{Μικρό πλάτος \(A_0\)}: Όταν το σήμα εισόδου έχει πολύ μικρή ενέργεια, το εκκρεμές δεν διεγείρεται επαρκώς, με συνέπεια τα σφάλματα εκτίμησης των παραμέτρων να είναι αυξημένα.
    \item \textbf{Μεγαλύτερο πλάτος \(A_0\)}: Καθώς αυξάνεται το πλάτος, η ταλάντωση γίνεται πιο έντονη και τα δεδομένα δειγματοληψίας παρέχουν πλουσιότερη πληροφορία στο αλγόριθμο. Αυτό συνήθως οδηγεί σε \emph{μείωση} των σφαλμάτων, τουλάχιστον μέχρι να εμφανιστούν μη γραμμικά φαινόμενα ή κορεσμός.
    \item \textbf{Συμβιβασμός}: Ενώ η υψηλότερη διέγερση (\(A_0\) μεγάλο) βελτιώνει την ακρίβεια, στην πράξη μπορεί να επιφέρει αυξημένες δυνάμεις/ροπές στο φυσικό σύστημα, ίσως ανεπιθύμητες. Έτσι επιλέγεται μια τιμή \(A_0\) επαρκώς μεγάλη ώστε να αποφεύγεται η χαμηλή διέγερση, αλλά όχι τόσο μεγάλη ώστε να τίθεται σε κίνδυνο η ασφάλεια του συστήματος ή η γραμμικότητα του μοντέλου.
    \item \textbf{Σημείωση}: Ωστόσο όπως εύκολα παρατηρείται και στα διαγράμματα σφάλματος των \(m\), \(L\), \(c\) τα σφάλματα φαίνεται να είναι ιδιαίτερα μικρά και στις τρείς περιπτώσεις για κάθε \(A_0\).
\end{itemize}

\subsection*{Συμπέρασμα}

Η μελέτη του Θέματος~3(γ) δείχνει ότι το \emph{πλάτος} της εισόδου \(\sin(\omega t)\) διαδραματίζει έναν ρόλο στην ποιότητα της εκτίμησης των παραμέτρων. Ένα πολύ μικρό \(A_0\) δυσχεραίνει ελαφρώς την αναγνώριση, ενώ ένα αρκετά μεγαλύτερο \(A_0\) βελτιώνει τα αποτελέσματα. Στην πράξη, επιλέγεται ενδιάμεση τιμή του \(A_0\) που προσφέρει επαρκή διέγερση του συστήματος χωρίς να εισάγει μη επιθυμητές μη γραμμικότητες ή υψηλές καταπονήσεις. Ωστόσο οι αποκλίσεις του σφάλματος σε κάθε περίπτωση δεν είναι ιδιαίτερα αισθητές (μικρό σφάλμα σε κάθε περίπτωση). 


\bibliographystyle{plain}
\begin{thebibliography}{1}
    \bibitem{lnmpikas}
    \en https://elearning.auth.gr/course/view.php?id=16318
\end{thebibliography}

\end{document}